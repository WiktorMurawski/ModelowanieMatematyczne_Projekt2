\documentclass[a4paper, 12pt, twoside, openany]{article}

% Pakiety
\usepackage[utf8]{inputenc}
\usepackage[T1]{fontenc}
\usepackage[polish]{babel}
\usepackage[hidelinks]{hyperref} % Hiperłącza bez ramek
%\usepackage{hyperref} % Hiperłącza
\usepackage{amsmath, amssymb} % Pakiety matematyczne
\usepackage{mathtools}
\usepackage{graphicx} % Obsługa grafiki
\graphicspath{{../images/}}
\usepackage{enumitem}
%\usepackage{fontspec}
\usepackage{float}
\usepackage[font=footnotesize,labelfont=bf]{caption}
\usepackage{setspace}
\usepackage{geometry} % Ustawienia marginesów
\geometry{
	inner=20mm, % margines wewnętrzny
	outer=20mm, % margines zewnętrzny
	top=25mm,   % margines górny
	bottom=25mm % margines dolny
}
\usepackage{xcolor}   % Pakiet do kolorów
\usepackage{listingsutf8} % Pakiet do listingu kodu
\lstdefinestyle{mystyle}{
	inputencoding=utf8,           % Kodowanie UTF-8
	extendedchars=true,           % Obsługa znaków spoza ASCII
	basicstyle=\ttfamily\footnotesize,   % Styl podstawowy
	language=Matlab,              % Język kodu
	keywordstyle=\bfseries\color{blue}, % Styl słów kluczowych
	commentstyle=\itshape\color{green!50!black}, % Styl komentarzy
	stringstyle=\color{red},      % Styl tekstu w cudzysłowach
	numbers=left,                 % Numeracja linii po lewej stronie
	numberstyle=\color{gray},     % Styl numerów linii
	frame=single,                 % Ramka wokół kodu
	breaklines=false,             % Zawijanie linii
	backgroundcolor=\color{gray!10}, % Kolor tła
	tabsize=2,                    % Wielkość tabulacji
	showstringspaces=false,        % Ukrywanie spacji w ciągach tekstowych
	literate={ą}{{\k{a}}}1
	{Ą}{{\k{A}}}1
	{ć}{{\'{c}}}1
	{Ć}{{\'{C}}}1
	{ę}{{\k{e}}}1
	{Ę}{{\k{E}}}1
	{ł}{\l}1
	{Ł}{\L}1
	{ń}{{\'{n}}}1
	{Ń}{{\'{N}}}1
	{ó}{{\'{o}}}1
	{Ó}{{\'{O}}}1
	{ś}{{\'{s}}}1
	{Ś}{{\'{S}}}1
	{ź}{{\'{z}}}1
	{Ź}{{\'{Z}}}1
	{ż}{{\.{z}}}1
	{Ż}{{\.{Z}}}1
}
\lstset{style=mystyle}

\renewcommand{\figurename}{Rys}

\newcommand{\y}{\mathbf{y}}
\newcommand{\I}{\mathbf{I}}
\newcommand{\A}{\mathbf{A}}
\newcommand{\f}{\mathbf{f}}
\renewcommand{\b}{\mathbf{b}}

\newcommand{\tytul}{Estymacja parametrów modelu}
\newcommand{\autor}{Wiktor Murawski}
\newcommand{\uczelnia}{Politechnika Warszawska}
\newcommand{\wydzial}{Wydział Matematyki i Nauk Informacyjnych}
\newcommand{\prowadzacy}{dr inż. Jakub Wagner}
\newcommand{\przedmiot}{Modelowanie matematyczne}
\newcommand{\miejsce}{Warszawa}
\date{\today}

% Dokument
\begin{document}
	
	% STRONA TYTUŁOWA
	\begin{titlepage}
		\centering
		\vspace*{1cm}
		\LARGE\textbf \tytul \\
		\vspace{1.5cm}
		\large
		%\normalsize
		Autor: \autor \\
		\vspace{1cm}
		Przedmiot: \przedmiot \\
		Prowadzący: \prowadzacy \\
		\vspace{2cm}
		\uczelnia \\
		\wydzial \\
		\vspace{2cm}
		Oświadczam, że niniejsza praca, stanowiąca podstawę do uznania osiągnięcia efektów
		uczenia się z przedmiotu Modelowanie matematyczne, została wykonana przeze mnie samodzielnie.\\
		\vspace{2cm}
		\miejsce \\
		\today \\
	\end{titlepage}
	
	% SPIS TREŚCI
	\tableofcontents
	\newpage
	
	% Lista symboli i akronimów
	\section{Lista Symboli i Akronimów}
	%\addcontentsline{toc}{section}{Lista Symboli i Akronimów}
	\begin{spacing}{1.5}
		\begin{tabbing}
			\hspace{5cm} \= \hspace{10cm} \= \kill
			\text{URRZ} \> układ równań różniczkowych zwyczajnych \\
			$t$ \> czas \\
			$x_k(t)$ i $y_k(t)$ \> współrzędne położenia $k$-tego obiektu \\
			$m$ \> masa obiektu \\
			$G$ \> stała grawitacyjna \\
			$r_{jk}(t)$ \> odległość pomiędzy obiektem $j$ a obiektem $k$ dla $j,k = 1,2,3$.
		\end{tabbing}
	\end{spacing}
	\newpage
	
	% Wprowadzenie
	\section{Wprowadzenie}
	Dane zawarte w pliku \texttt{data\_30.csv} reprezentują wyniki pomiaru położenia trzech obiektów o
	identycznych masach $m$, przyciągających się grawitacyjnie. Trajektorie ruchu tych obiektów opisane
	są następującym układem nieliniowych równań różniczkowych zwyczajnych drugiego rzędu:
	\[\hspace{-5cm}
	\left\{
	\begin{alignedat}{4}
		\dfrac{d^2x_1(t)}{dt^2} & = Gm \Big( & \dfrac{x_2(t) - x_1(t)}{r^3_{12}(t)} & + \dfrac{x_3(t) - x_1(t)}{r^3_{13}(t)} & \Big) \\
		\dfrac{d^2y_1(t)}{dt^2} & = Gm \Big( & \dfrac{y_2(t) - y_1(t)}{r^3_{12}(t)} & + \dfrac{y_3(t) - y_1(t)}{r^3_{13}(t)} & \Big) \\
		\dfrac{d^2x_2(t)}{dt^2} & = Gm \Big( & \dfrac{x_3(t) - x_2(t)}{r^3_{23}(t)} & + \dfrac{x_1(t) - x_2(t)}{r^3_{12}(t)} & \Big) \\
		\dfrac{d^2y_2(t)}{dt^2} & = Gm \Big( & \dfrac{y_3(t) - y_2(t)}{r^3_{23}(t)} & + \dfrac{y_1(t) - y_2(t)}{r^3_{12}(t)} & \Big) \\
		\dfrac{d^2x_3(t)}{dt^2} & = Gm \Big( & \dfrac{x_1(t) - x_3(t)}{r^3_{13}(t)} & + \dfrac{x_2(t) - x_3(t)}{r^3_{23}(t)} & \Big) \\
		\dfrac{d^2y_3(t)}{dt^2} & = Gm \Big( & \dfrac{y_1(t) - y_3(t)}{r^3_{13}(t)} & + \dfrac{y_2(t) - y_3(t)}{r^3_{23}(t)} & \Big)
	\end{alignedat}
	\right.
	\]
	gdzie:
	\begin{itemize}[label=\footnotesize$\bullet$, topsep=0pt, parsep=0pt, leftmargin=10mm]
		\item $t$ oznacza czas,
		\item $x_k(t)$ i $y_k(t)$ to współrzędne położenia $k$-tego obiektu dla $k = 1,2,3$,
		\item $m$ to masa obiektu,
		\item $G$ to stała grawitacyjna,
		\item $r_{jk}(t) \equiv \sqrt{\left[x_k(t) - x_j(t)\right]^2 + \left[y_k(t) - y_j(t)\right]^2}$ dla $j,k = 1,2,3$.
	\end{itemize}
	
	%\newpage
	
	% Metodyka i wyniki doświadczeń
	\section{Metodyka i Wyniki Doświadczeń}
	%Opis wykonanych doświadczeń i obliczeń, zarówno w środowisku MATLAB, jak i na papierze. Szczegóły pozwalające na odtworzenie wyników.
	
	%\newpage
	
	% Dyskusja wyników eksperymentów numerycznych
	\section{Dyskusja Wyników Eksperymentów Numerycznych}
	
	\newpage
	
	% Lista źródeł informacji
	\section*{Bibliografia}
	\addcontentsline{toc}{section}{Bibliografia}
	\begin{enumerate}
		\item Dokumentacja MATLAB: \url{https://www.mathworks.com/help/matlab}.
	\end{enumerate}
	
	\newpage
	
	% Listing opracowanych programów
	\section*{Listing Programów}
	\addcontentsline{toc}{section}{Listing Programów}
	
	\subsection*{plik: \texttt{Projekt2.m}}\vspace{-0.5em}
	\lstinputlisting{../Projekt2.m}
	\subsection*{plik: \texttt{ApproximateDerivative.m}}\vspace{-0.5em}
	\lstinputlisting{../ApproximateDerivative.m}
	\subsection*{plik: \texttt{ODEFunction.m}}\vspace{-0.5em}
	\lstinputlisting{../ODEFunction.m}
	\subsection*{plik: \texttt{Criterium.m}}\vspace{-0.5em}
	\lstinputlisting{../Criterium.m}
	\subsection*{plik: \texttt{ApproximateMass.m}}\vspace{-0.5em}
	\lstinputlisting{../ApproximateMass.m}
	\subsection*{plik: \texttt{Visualize.m}}\vspace{-0.5em}
	\lstinputlisting{../Visualize.m}

\end{document}
